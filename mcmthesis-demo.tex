%%
%% This is file `mcmthesis-demo.tex',
%% generated with the docstrip utility.
%%
%% The original source files were:
%%
%% mcmthesis.dtx  (with options: `demo')
%% !Mode:: "TeX:UTF-8"
%% -----------------------------------
%%
%% This is a generated file.
%%
%% Copyright (C)
%%     2010 -- 2015 by latexstudio
%%     2014 -- 2016 by Liam Huang
%%     2014 -- 2016 by latexstudio.net
%%
%% This work may be distributed and/or modified under the
%% conditions of the LaTeX Project Public License, either version 1.3
%% of this license or (at your option) any later version.
%% The latest version of this license is in
%%   http://www.latex-project.org/lppl.txt
%% and version 1.3 or later is part of all distributions of LaTeX
%% version 2005/12/01 or later.
%%
%% This work has the LPPL maintenance status `maintained'.
%%
%% The Current Maintainer of this work is Liam Huang.
%%
\documentclass{mcmthesis}
\mcmsetup{CTeX = true,   % 使用 CTeX 套装时,设置为 true
        tcn = 0000, problem = A,% 队伍控制号码,接受一个字符串作为值;选题,接受一个字符串作为值;
        sheet = true, %为真时将输出摘要页,否则不输出;默认为 true。
        color = red,  %设置控制页的题目号的颜色
        titleinsheet = true, %为真时将在摘要页输出标题,否则不输出;默认为 false。
        keywordsinsheet = true,%为真时将在摘要页输出关键字,否则不输出;默认为 false。
        titlepage = false,%为真时将输出标题页,否则不输出;默认为 true。
        abstract = true}%为真时将在标题页输出摘要和关键词,否则不输出;默认值为 true。
\usepackage{palatino}  %控制正文字体,若是不喜欢可以注释掉。
\usepackage{lipsum}
\title{The \LaTeX{} Template for MCM Version \MCMversion}
\author{\url{http://www.latexstudio.net}\\[3pt]  \href{http://www.latexstudio.net/}
  {\includegraphics[width=7cm]{mcmthesis-logo}}}
\date{\today}

\makeatletter
\renewcommand*\l@section{\@dottedtocline{1}{12pt}{12pt}}
\makeatother

\begin{document}
\begin{abstract}
The Abstract should be
structured to include the following details: Background,
the context and purpose of the study; Results, the main
findings; Conclusions, brief summary and potential im-
plications. Please minimize the use of abbreviations and
do not cite references in the abstract.
\begin{keywords}
keyword1; keyword2
\end{keywords}
\end{abstract}
\maketitle

\tableofcontents


\section{Introduction}
\subsection{Problem Background}
\par As narcotic analgesics, opioids are primarily used for pain relief and anesthesia to reduce patients' suffering, which contribute a lot to modern medicine since its appearance. However, it also has many side effects, including itchiness, nausea, respiratory depression, constipation, and, what counts most, addiction. Considering the adverse effects above, we tend to enforce the regulation of opioids to reduce the risk of addiction.

Nowadays, The United States is undergoing one of its worst drug crises caused by opioids. Hundreds of people die from opioid-related overdoses while more and more Americans are suffering from opioid addiction. The crisis has reached such a scale that, beyond the risks it brings to public health, the loss of people with advanced degrees will drag the development of economy and national security.

With the data offered by Drug Enforcement Administration's Office of Diversion Control about the situation of drug abusing in five states, we can know more about the characteristics of opioid epidemic and find out the factors contributing to the growth in opioid use and addiction, so that we can set a therapy to alleviate the problem.
\subsection{View of Our Work}



\subsection{Assumptions}

Symbols

\begin{tabular}{cp{0.6\textwidth}}
\toprule
 Symbols & Description\\
\midrule
 $x$ & position \\
 $v$ & velocity \\
 $a$ & acceleration \\
 $t$ & time \\
 $F$ & force\\
\bottomrule
\end{tabular}



\lipsum[7]

\section{Analysis of the Problem}
\begin{figure}[h]
\small
\centering
\includegraphics[width=12cm]{mcmthesis-aaa.eps}
\caption{aa} \label{fig:aa}
\end{figure}

\lipsum[8] \eqref{aa}
\begin{equation}
a^2 \label{aa}
\end{equation}

\[
  \begin{pmatrix}{*{20}c}
  {a_{11} } & {a_{12} } & {a_{13} }  \\
  {a_{21} } & {a_{22} } & {a_{23} }  \\
  {a_{31} } & {a_{32} } & {a_{33} }  \\
  \end{pmatrix}
  = \frac{{Opposite}}{{Hypotenuse}}\cos ^{ - 1} \theta \arcsin \theta
\]
\lipsum[9]

\[
  p_{j}=\begin{cases} 0,&\text{if $j$ is odd}\\
  r!\,(-1)^{j/2},&\text{if $j$ is even}
  \end{cases}
\]

\lipsum[10]

\[
  \arcsin \theta  =
  \mathop{{\int\!\!\!\!\!\int\!\!\!\!\!\int}\mkern-31.2mu
  \bigodot}\limits_\varphi
  {\mathop {\lim }\limits_{x \to \infty } \frac{{n!}}{{r!\left( {n - r}
  \right)!}}} \eqno (1)
\]

\section{Calculating and Simplifying the Model  }
\lipsum[11]

\section{The Model Results}
\lipsum[6]

\section{Validating the Model}
\lipsum[9]

\section{Conclusions}
\lipsum[6]

\section{A Summary}
\lipsum[6]

\section{Evaluate of the Mode}

\section{Strengths and weaknesses}
\lipsum[12]

\subsection{Strengths}
\begin{itemize}
\item \textbf{Applies widely}\\
This  system can be used for many types of airplanes, and it also
solves the interference during  the procedure of the boarding
airplane,as described above we can get to the  optimization
boarding time.We also know that all the service is automate.
\item \textbf{Improve the quality of the airport service}\\
Balancing the cost of the cost and the benefit, it will bring in
more convenient  for airport and passengers.It also saves many
human resources for the airline. \item \textbf{}
\end{itemize}

\begin{thebibliography}{99}
\bibitem{1} D.~E. KNUTH   The \TeX{}book  the American
Mathematical Society and Addison-Wesley
Publishing Company , 1984-1986.
\bibitem{2}Lamport, Leslie,  \LaTeX{}: `` A Document Preparation System '',
Addison-Wesley Publishing Company, 1986.
\bibitem{3}\url{http://www.latexstudio.net/}
\bibitem{4}\url{http://www.chinatex.org/}
\end{thebibliography}

\begin{appendices}

\section{First appendix}

\lipsum[13]

Here are simulation programmes we used in our model as follow.\\

\textbf{\textcolor[rgb]{0.98,0.00,0.00}{Input matlab source:}}
\lstinputlisting[language=Matlab]{./code/mcmthesis-matlab1.m}

\section{Second appendix}

some more text \textcolor[rgb]{0.98,0.00,0.00}{\textbf{Input C++ source:}}
\lstinputlisting[language=C++]{./code/mcmthesis-sudoku.cpp}

\end{appendices}
\end{document}

%%
%% This work consists of these files mcmthesis.dtx,
%%                                   figures/ and
%%                                   code/,
%% and the derived files             mcmthesis.cls,
%%                                   mcmthesis-demo.tex,
%%                                   README,
%%                                   LICENSE,
%%                                   mcmthesis.pdf and
%%                                   mcmthesis-demo.pdf.
%%
%% End of file `mcmthesis-demo.tex'.
